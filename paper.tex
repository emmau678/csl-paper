\def\paperversiondraft{draft}
\def\paperversionnormal{normal}

% If the paper version is set to 'normal' mode keep it,
% otherwise set it to 'draft' mode.
\ifx\paperversion\paperversionnormal
\else
  \def\paperversion{draft}
\fi

\documentclass[review, anonymous, acmsmall]{acmart}

\def\acmversionanonymous{anonymous}
\def\acmversionjournal{journal}
\def\acmversionnone{none}

\makeatletter
\if@ACM@anonymous
  \def\acmversion{anonymous}
\else
  \def\acmversion{journal}
\fi
\makeatother

\usepackage{colortbl}

% 'draftonly' environment
\usepackage{environ}
\ifx\paperversion\paperversiondraft
\newenvironment{draftonly}{}{}
\else
\NewEnviron{draftonly}{}
\fi

% Most PL conferences are edited by conference-publishing.com. Follow their
% advice to add the following packages.
%
% The first enables the use of UTF-8 as character encoding, which is the
% standard nowadays. The second ensures the use of font encodings that support
% accented characters etc. (Why should I use this?). The mictotype package
% enables certain features 'to­wards ty­po­graph­i­cal per­fec­tion
\usepackage[utf8]{inputenc}
\usepackage[T1]{fontenc}
\usepackage{microtype}

\usepackage{xargs}
\usepackage{lipsum}
\usepackage{xparse}
\usepackage{xifthen, xstring}
\usepackage{xspace}
\usepackage{marginnote}
\usepackage{etoolbox}
\usepackage[acronym,shortcuts]{glossaries}
\usepackage{amsmath}
\usepackage{thmtools} % required for autoref to lemmas
\usepackage{algorithm}
\usepackage[noend]{algpseudocode}
\usepackage{hyphenat}
\usepackage[shortcuts]{extdash}

%%%%%%%%%%%%%%%%%%%%%%%%%%%%%%%%%%%%%%%%%%%%%%%%%%%%%%%%%%%%%%%%%%%%%%%%%%%%%%%
% Broaden margins to make room for todo notes
%%%%%%%%%%%%%%%%%%%%%%%%%%%%%%%%%%%%%%%%%%%%%%%%%%%%%%%%%%%%%%%%%%%%%%%%%%%%%%%

\makeatletter
\patchcmd{\@addmarginpar}{\ifodd\c@page}{\ifodd\c@page\@tempcnta\m@ne}{}{}
\makeatother
\ifx\paperversion\paperversiondraft
  \makeatletter
  \if@ACM@journal
    \geometry{asymmetric}
    \paperwidth=\dimexpr \paperwidth + 3.5cm\relax
    \oddsidemargin=\dimexpr\oddsidemargin + 0cm\relax
    \evensidemargin=\dimexpr\evensidemargin + 0cm\relax
    \marginparwidth=\dimexpr \marginparwidth + 3cm\relax
    \setlength{\marginparwidth}{4.6cm}
    % This makeatletter box helps to move notes to the right
    \makeatletter
    \long\def\@mn@@@marginnote[#1]#2[#3]{%
      \begingroup
        \ifmmode\mn@strut\let\@tempa\mn@vadjust\else
          \if@inlabel\leavevmode\fi
          \ifhmode\mn@strut\let\@tempa\mn@vadjust\else\let\@tempa\mn@vlap\fi
        \fi
        \@tempa{%
          \vbox to\z@{%
            \vss
            \@mn@margintest
            \if@reversemargin\if@tempswa
                \@tempswafalse
              \else
                \@tempswatrue
            \fi\fi
            %\if@tempswa
              \rlap{%
                \if@mn@verbose
                  \PackageInfo{marginnote}{xpos seems to be \@mn@currxpos}%
                \fi
                \begingroup
                  \ifx\@mn@currxpos\relax\else\ifx\@mn@currxpos\@empty\else
                      \kern-\dimexpr\@mn@currxpos\relax
                  \fi\fi
                  \ifx\@mn@currpage\relax
                    \let\@mn@currpage\@ne
                  \fi
                  \if@twoside\ifodd\@mn@currpage\relax
                      \kern\oddsidemargin
                    \else
                      \kern\evensidemargin
                    \fi
                  \else
                    \kern\oddsidemargin
                  \fi
                  \kern 1in
                \endgroup
                \kern\marginnotetextwidth\kern\marginparsep
                \vbox to\z@{\kern\marginnotevadjust\kern #3
                  \vbox to\z@{%
                    \hsize\marginparwidth
                    \linewidth\hsize
                    \kern-\parskip
                    \marginfont\raggedrightmarginnote\strut\hspace{\z@}%
                    \ignorespaces#2\endgraf
                    \vss}%
                  \vss}%
              }%
          }%
        }%
      \endgroup
    }
    \makeatother
  \else
    \paperwidth=\dimexpr \paperwidth + 6cm\relax
    \oddsidemargin=\dimexpr\oddsidemargin + 3cm\relax
    \evensidemargin=\dimexpr\evensidemargin + 3cm\relax
    \marginparwidth=\dimexpr \marginparwidth + 3cm\relax
    \setlength{\marginparwidth}{4.6cm}
  \fi
  \makeatother
\fi

%%%%%%%%%%%%%%%%%%%%%%%%%%%%%%%%%%%%%%%%%%%%%%%%%%%%%%%%%%%%%%%%%%%%%%%%%%%%%%%
% Add \createtodoauthor command
%%%%%%%%%%%%%%%%%%%%%%%%%%%%%%%%%%%%%%%%%%%%%%%%%%%%%%%%%%%%%%%%%%%%%%%%%%%%%%%

\usepackage[textsize=tiny]{todonotes}
\usepackage[normalem]{ulem}

\makeatletter
\font\uwavefont=lasyb10 scaled 652
\newcommand\colorwave[1][blue]{\bgroup\markoverwith{\lower3\p@\hbox{\uwavefont\textcolor{#1}{\char58}}}\ULon}

\newcommand\highlight[2]{{\color{#1}{\colorwave[#1]{#2}}}}
\makeatother

\makeatletter
\newcommand\InFloat[2]{\ifnum\@floatpenalty<0\relax#1\else#2\fi}
\makeatother

\newboolean{inComment}
\setboolean{inComment}{false}

\ifx\paperversion\paperversiondraft
\newcommand\createtodoauthor[2]{
  \def\tmpdefault{emptystring}
  \expandafter\newcommand\csname #1\endcsname[2][\tmpdefault]{% comment to avoid spurious whitespace
    \ifthenelse{\boolean{inComment}}{% comment to avoid spurious whitespace
      \PackageError{paper-template}{Comments in comments not supported}{}% comment to avoid spurious whitespace
    }{}\setboolean{inComment}{true}% comment to avoid spurious whitespace
    \def\tmp{##1}% comment to avoid spurious whitespace
    \InFloat{
        \smash{
	  \marginnote{
	    \todo[inline,linecolor=#2,backgroundcolor=#2,bordercolor=#2]
	      {\textbf{#1 (Figure):} ##2}
          }
        }
    }{\ifthenelse{\equal{\tmp}{\tmpdefault}} % Is there text to highlight?
      {\todo[linecolor=#2,backgroundcolor=#2,bordercolor=#2]{\textbf{#1:} ##2}\ignorespaces}
      {\ifthenelse{\equal{##2}{}} % Is there a note?
        {\highlight{#2}{##1}}
        {\highlight{#2}{##1}\todo[linecolor=#2,backgroundcolor=#2,bordercolor=#2]
	  {\textbf{#1:} ##2}% comment to avoid spurious whitespace
	}% comment to avoid spurious whitespace
      }% comment to avoid spurious whitespace
    }% comment to avoid spurious whitespace
    \setboolean{inComment}{false}% comment to avoid spurious whitespace
  }
}
\else
\newcommand\createtodoauthor[2]{%
\expandafter\newcommand\csname #1\endcsname[2][]{##1}%
}%
\fi

%%%%%%%%%%%%%%%%%%%%%%%%%%%%%%%%%%%%%%%%%%%%%%%%%%%%%%%%%%%%%%%%%%%%%%%%%%%%%%%
% Add minted and support custom lexers
%%%%%%%%%%%%%%%%%%%%%%%%%%%%%%%%%%%%%%%%%%%%%%%%%%%%%%%%%%%%%%%%%%%%%%%%%%%%%%%
\usepackage{minted}
\usepackage{etoolbox}

\makeatletter
\ifcsdef{minted@optlistcl@quote}
{
\ifwindows
  \renewcommand{\minted@optlistcl@quote}[2]{%
    \ifstrempty{#2}{\detokenize{#1}}{\detokenize{#1="#2"}}}
\else
  \renewcommand{\minted@optlistcl@quote}[2]{%
    \ifstrempty{#2}{\detokenize{#1}}{\detokenize{#1='#2'}}}
\fi

% similar to \minted@def@optcl@switch
\newcommand{\minted@def@optcl@novalue}[2]{%
  \define@booleankey{minted@opt@g}{#1}%
    {\minted@addto@optlistcl{\minted@optlistcl@g}{#2}{}%
     \@namedef{minted@opt@g:#1}{true}}
    {\@namedef{minted@opt@g:#1}{false}}
  \define@booleankey{minted@opt@g@i}{#1}%
    {\minted@addto@optlistcl{\minted@optlistcl@g@i}{#2}{}%
     \@namedef{minted@opt@g@i:#1}{true}}
    {\@namedef{minted@opt@g@i:#1}{false}}
  \define@booleankey{minted@opt@lang}{#1}%
    {\minted@addto@optlistcl@lang{minted@optlistcl@lang\minted@lang}{#2}{}%
     \@namedef{minted@opt@lang\minted@lang:#1}{true}}
    {\@namedef{minted@opt@lang\minted@lang:#1}{false}}
  \define@booleankey{minted@opt@lang@i}{#1}%
    {\minted@addto@optlistcl@lang{minted@optlistcl@lang\minted@lang @i}{#2}{}%
     \@namedef{minted@opt@lang\minted@lang @i:#1}{true}}
    {\@namedef{minted@opt@lang\minted@lang @i:#1}{false}}
  \define@booleankey{minted@opt@cmd}{#1}%
      {\minted@addto@optlistcl{\minted@optlistcl@cmd}{#2}{}%
        \@namedef{minted@opt@cmd:#1}{true}}
      {\@namedef{minted@opt@cmd:#1}{false}}
}
\minted@def@optcl@novalue{customlexer}{-x}
}
{
}
\makeatother

\usepackage{tikz}
\usetikzlibrary{arrows}
\usetikzlibrary{shapes}

%%%%%%%%%%%%%%%%%%%%%%%%%%%%%%%%%%%%%%%%%%%%%%%%%%%%%%%%%%%%%%%%%%%%%%%%%%%%%%%
% Base style and command for \circled to print a colored circle
%%%%%%%%%%%%%%%%%%%%%%%%%%%%%%%%%%%%%%%%%%%%%%%%%%%%%%%%%%%%%%%%%%%%%%%%%%%%%%%
% Width is assured to be the same across all characters using the typewriter font which is monospaced
% Zeroing out the inner separator removes padding between content and node (inner sep).
% Zeroing out the outer separator removes space between the node border and its anchors (e.g., east).
% Minimum size was derived on experimentation and it may need adjustment when changing font style/size.
%
% There is no guarantee for the letter ascenders/descenders to baseline when set to char.base, hence adding \strut.
% which is an invisible vertical rule with the height and depth of the parentheses ( and ).
% It ensures that the line height in a line of text is at least as large as if it contained parentheses. 
\tikzset{
  circledstyle/.style={
    shape=circle,
    #1,
    font=\tt\small,
    inner sep=0pt,
    outer sep=0pt,
    minimum size=1.2em,
    text=black
  }
}

% define a base tikz node for circled commands accepting a fill colour and the node text as arguments
\DeclareRobustCommand{\circledbase}[3][]{%
    \tikz[baseline=(char.base)]{\node[circledstyle, fill=#2] (char) {#3\strut};}%
}


\input{tex/acm.tex}

\usemintedstyle{colorful}

% Newer versions of minted require the 'customlexer' argument for custom lexers
% whereas older versions require the '-x' to be passed via the command line.
\makeatletter
\ifcsdef{MintedExecutable}
{
  % minted v3
  \newminted[mlir]{tools/lexers/MLIRLexer.py:MLIRLexerOnlyOps}{mathescape}
  \newminted[xdsl]{tools/lexers/MLIRLexer.py:MLIRLexer}{mathescape, style=murphy}
  \newminted[lean4]{tools/lexers/Lean4Lexer.py:Lean4Lexer}{mathescape}
}
{
  \ifcsdef{minted@optlistcl@quote}
  {
    \newminted[mlir]{tools/lexers/MLIRLexer.py:MLIRLexerOnlyOps}{customlexer, mathescape}
    \newminted[xdsl]{tools/lexers/MLIRLexer.py:MLIRLexer}{customlexer, mathescape, style=murphy}
    \newminted[lean4]{tools/lexers/Lean4Lexer.py:Lean4Lexer}{customlexer, mathescape}
  }
  {
    \newminted[mlir]{tools/lexers/MLIRLexer.py:MLIRLexerOnlyOps -x}{mathescape}
    \newminted[xdsl]{tools/lexers/MLIRLexer.py:MLIRLexer -x}{mathescape, style=murphy}
    \newminted[lean4]{tools/lexers/Lean4Lexer.py:Lean4Lexer -x}{mathescape}
  }
}
\makeatother

% We use the following color scheme
%
% This scheme is both print-friendly and colorblind safe for
% up to four colors (including the red tones makes it not
% colorblind safe any more)
%
% https://colorbrewer2.org/#type=qualitative&scheme=Paired&n=4

\definecolor{pairedNegOneLightGray}{HTML}{cacaca}
\definecolor{pairedNegTwoDarkGray}{HTML}{827b7b}
\definecolor{pairedOneLightBlue}{HTML}{a6cee3}
\definecolor{pairedTwoDarkBlue}{HTML}{1f78b4}
\definecolor{pairedThreeLightGreen}{HTML}{b2df8a}
\definecolor{pairedFourDarkGreen}{HTML}{33a02c}
\definecolor{pairedFiveLightRed}{HTML}{fb9a99}
\definecolor{pairedSixDarkRed}{HTML}{e31a1c}

\createtodoauthor{grosser}{pairedOneLightBlue}
\createtodoauthor{authorTwo}{pairedTwoDarkBlue}
\createtodoauthor{authorThree}{pairedThreeLightGreen}
\createtodoauthor{authorFour}{pairedFourDarkGreen}
\createtodoauthor{authorFive}{pairedFiveLightRed}
\createtodoauthor{authorSix}{pairedSixDarkRed}

\newacronym{ir}{IR}{Intermediate Representation}

\graphicspath{{./images/}}

% Define macros that are used in this paper
%
% We require all macros to end with a delimiter (by default {}) to enusure
% that LaTeX adds whitespace correctly.
\makeatletter
\newcommand\requiredelimiter[2][########]{%
  \ifdefined#2%
    \def\@temp{\def#2#1}%
    \expandafter\@temp\expandafter{#2}%
  \else
    \@latex@error{\noexpand#2undefined}\@ehc
  \fi
}
\@onlypreamble\requiredelimiter
\makeatother

\newcommand\newdelimitedcommand[2]{
\expandafter\newcommand\csname #1\endcsname{#2}
\expandafter\requiredelimiter
\csname #1 \endcsname
}

\newdelimitedcommand{toolname}{Tool}

\usepackage{booktabs}
\newcommand{\ra}[1]{\renewcommand{\arraystretch}{#1}}

\usepackage[verbose]{newunicodechar}
\newunicodechar{₁}{\ensuremath{_1}}
\newunicodechar{₂}{\ensuremath{_2}}
\newunicodechar{∀}{\ensuremath{\forall}}
\newunicodechar{α}{\ensuremath{\alpha}}
\newunicodechar{β}{\ensuremath{\beta}}

% \circled command to print a colored circle.
% \circled{1} pretty-prints "(1)"
% This is useful to refer to labels that are embedded within figures.
\DeclareRobustCommand{\circled}[2][]{%
    \ifthenelse{\isempty{#1}}%
        {\circledbase{pairedOneLightBlue}{#2}}%
        {\autoref{#1}: \hyperref[#1]{\circledbase{pairedOneLightBlue}{#2}}}%
}

% listings don't write "Listing" in autoref without this.
\providecommand*{\listingautorefname}{Listing}
\renewcommand{\sectionautorefname}{Section}
\renewcommand{\subsectionautorefname}{Section}
\renewcommand{\subsubsectionautorefname}{Section}

\begin{document}

%% Title information
\title[Lagrangian Particle Flows at the Wafer Scale]{Unleashing Spatial Parallelism for\\Lagrangian Particle Flows at the Wafer Scale}



%% Author information
%% Contents and number of authors suppressed with 'anonymous'.
%% Each author should be introduced by \author, followed by
%% \authornote (optional), \orcid (optional), \affiliation, and
%% \email.
%% An author may have multiple affiliations and/or emails; repeat the
%% appropriate command.
%% Many elements are not rendered, but should be provided for metadata
%% extraction tools.
\author{First1 Last1}
\authornote{with author1 note}          %% \authornote is optional;
                                      %% can be repeated if necessary
\orcid{nnnn-nnnn-nnnn-nnnn}             %% \orcid is optional
\affiliation{
  \position{Position1}
  \department{Department1}              %% \department is recommended
  \institution{Institution1}            %% \institution is required
  \streetaddress{Street1 Address1}
  \city{City1}
  \state{State1}
  \postcode{Post-Code1}
  \country{Country1}
}
\email{first1.last1@inst1.edu}          %% \email is recommended

\author{First2 Last2}
\authornote{with author2 note}          %% \authornote is optional;
                                      %% can be repeated if necessary
\orcid{nnnn-nnnn-nnnn-nnnn}             %% \orcid is optional
\affiliation{
  \position{Position2a}
  \department{Department2a}             %% \department is recommended
  \institution{Institution2a}           %% \institution is required
  \streetaddress{Street2a Address2a}
  \city{City2a}
  \state{State2a}
  \postcode{Post-Code2a}
  \country{Country2a}
}
\email{first2.last2@inst2a.com}         %% \email is recommended
\affiliation{
  \position{Position2b}
  \department{Department2b}             %% \department is recommended
  \institution{Institution2b}           %% \institution is required
  \streetaddress{Street3b Address2b}
  \city{City2b}
  \state{State2b}
  \postcode{Post-Code2b}
  \country{Country2b}
}
\email{first2.last2@inst2b.org}         %% \email is recommended

\begin{abstract}
% An abstract should consist of six main sentences:
%  1. Introduction. In one sentence, what’s the topic?
%  2. State the problem you tackle.
%  3. Summarize (in one sentence) why nobody else has adequately answered the research question yet.
%  4. Explain, in one sentence, how you tackled the research question.
%  5. In one sentence, how did you go about doing the research that follows from your big idea.
%  6. As a single sentence, what’s the key impact of your research?

% (http://www.easterbrook.ca/steve/2010/01/how-to-write-a-scientific-abstract-in-six-easy-steps/)

  Particle-based simulations require high compute performance and bandwidth to simulate realistic workloads at meaningful scales and resolution.
  The Cerebras Wafer-Scale Engine (WSE), a dataflow architecture comprising 900,000 processing elements, presents an alternative acceleration opportunity to CPU/GPU approaches. 
  Its flat memory layout and high-speed fabric interconnect pose a distinct optimization challenge compared to conventional communication and hierarchical memory bottlenecks.
  While mesh-based simulation techniques are efficient, they struggle to capture dynamic phenomena such as turbulent, free-surface and multiphase flows.
  This is where particle methods offer superior physical realism, but at an inherently steep cost of tracking individual particles and interactions with their neighbors as they flow freely through space.
  The tension between the target application of dynamic, unstructured particle flows and the target architecture of the WSE's structured grid requires strategic architectural mapping.
  We leverage a spatial parallelization approach to translate physical space to the architecture of the WSE and transcend the constraints of 1:1 element-PE mappings. 
  Dynamic and flexible routing of particles across this spatial domain, combined with support for variable particle populations per PE enables us to handle rapid, turbulent and high-velocity particle movement. 
  Our technique achieves exceptional performance and efficiency, showcasing significant potential to revolutionize particle-based simulations at the wafer scale.
  
\end{abstract}

% Only add ACM notes and keywords in camera ready version
% Drop citations and footnotes in draft and blind mode.
\ifx\acmversion\acmversionanonymous
\settopmatter{printacmref=false} % Removes citation information below abstract
\renewcommand\footnotetextcopyrightpermission[1]{} % removes footnote with conference information in first column
\fi
\ifx\acmversion\acmversionjournal
%% 2012 ACM Computing Classification System (CSS) concepts
%% Generate at 'http://dl.acm.org/ccs/ccs.cfm'.
\begin{CCSXML}
<ccs2012>
<concept>
<concept_id>10011007.10011006.10011008</concept_id>
<concept_desc>Software and its engineering~General programming languages</concept_desc>
<concept_significance>500</concept_significance>
</concept>
<concept>
<concept_id>10003456.10003457.10003521.10003525</concept_id>
<concept_desc>Social and professional topics~History of programming languages</concept_desc>
<concept_significance>300</concept_significance>
</concept>
</ccs2012>
\end{CCSXML}

\ccsdesc[500]{Software and its engineering~General programming languages}
\ccsdesc[300]{Social and professional topics~History of programming languages}
%% End of generated code

%% Keywords
%% comma separated list
\keywords{keyword1, keyword2, keyword3}
\fi

%% \maketitle
%% Note: \maketitle command must come after title commands, author
%% commands, abstract environment, Computing Classification System
%% environment and commands, and keywords command.
\maketitle

\section{Introduction}
Particle flow methods have emerged as powerful techniques for modelling dynamic phenomena spanning multiple domains with high physical accuracy.
Meshfree, Lagrangian simulations allow precise motion tracking through continuous coordinate systems without lattice-imposed dicretization errors.
They offer inherent advatages over grid-based methods in handling high-velocity, multi-phase or free-surface flows and complex geometries.
By intrinsically capturing motion at the mesoscopic scale, they avoid surface approximations that can compromise the accuracy of grid-based approaches.

- Computational intensity and reasons 
- Architecture of Cerebras and why it's a good match
- Our approach (summarized)
- Contributions

\noindent
Our contributions are:
\begin{itemize}
	\item Contribution 1
	\item Contribution 2
	\item Contribution 3
	\item Contribution 4
\end{itemize}

\section{Wafer-Scale Spatial Parallelism for Lagrangian Dynamics}

\section{Background}
\subsection{The Cerebras WSE: A Massively Parallel Dataflow Architecture}
\subsection{Adapting HPC Workloads for Emerging Architectures}
\begin{itemize}
  \item limitations being reached by CPUs/GPUs
  \item greater power efficiency afforded by novel accelerators
  \item -> growing interest in repurposing accelerator hw for HPC applications
  \item FP32 precision as an inherent challenge with increasing research effort being invested in solutions
\end{itemize}
\subsection{Particle Simulations as a Target Application Domain}
\begin{itemize}
  \item Key algorithmic components / defining features of Lagrangian schemes
  \item How these apply to diverse applications of practical relevance
\end{itemize}

\section{Implementation}
\begin{itemize}
  \item Overview of the algorithmic components we have chosen to focus on and why (1/2 sentences) 
\end{itemize}
\subsection{Architectural Mapping and PE-local Co-ordinate Systems}
\subsection{Particle Data Representation and Memory Access Patterns}
\begin{itemize}
  \item include force kernel computation description
\end{itemize}
\subsection{Batched Communication Protocol for Variable-Length Buffers}
\begin{itemize}
  \item how this enables scalability
  \item how this eliminates redundant data communication
\end{itemize}
\subsection{Particle Migration and Dynamic Redistribution}
\begin{itemize}
  \item concise technical summary of the phase
  \item Steps:
  \begin{itemize}
    \item local rerouting
    \item initialization with sends
    \item dynamic rerouting and alignment priority
    \item support for dynamic-length sends (await receipt of any sent data)
  \end{itemize}
  \item Accompanying diagrams
\end{itemize}
\subsection{Neighbour Search and Interactions}
\begin{itemize}
  \item local interactions:
  \begin{itemize}
    \item while loops ensure no redundancy
    \item antisymmetry of kernels
  \end{itemize}
  \item dynamic interactions with received neighbor data
  \begin{itemize}
    \item efficiency of send/receive-all approach
    \item DSDs allow in-place updates
    \item routing scheme and accompanying diagram
  \end{itemize}
\end{itemize}

\section{Evaluation}
\begin{itemize}
  \item brief overview of what we evaluate.
\end{itemize}
\subsection{Experimental Setup}
\section{Evaluation}
\begin{itemize}
  \item Overview of what we evaluate
\end{itemize}
\subsection{Experimental Setup}
\subsection{Results}
\begin{itemize}
  \item Strong Scaling (perf and efficiency)
  \item Weak Scaling
\end{itemize}

\section{Limitations and Future Work}
\begin{itemize}
  \item FP32 precision
  \item Adaptive resolution and load balancing: how this will enable full usage of capacity instead of only permitting conservative measurements
  \item Practical applications
  \item Boundary conditions
\end{itemize}

\section{Related Work}

\section{Conclusion}

%% Acknowledgments
\begin{acks}                            %% acks environment is optional
                                        %% contents suppressed with 'anonymous'
  %% Commands \grantsponsor{<sponsorID>}{<name>}{<url>} and
  %% \grantnum[<url>]{<sponsorID>}{<number>} should be used to
  %% acknowledge financial support and will be used by metadata
  %% extraction tools.
  
\end{acks}

%% Bibliography
\bibliography{references}


%% Appendix
% Move \cleardouble and \appendix out of `draftonly` if you are using the
% appendix


\end{document}
